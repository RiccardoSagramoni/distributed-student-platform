In our web-application, four types of actors exist:
\begin{enumerate}
    \item Unregistered student
    \item Students
    \item Professors
    \item Admin
\end{enumerate}
\ \\
The operation that each kind of actor can perform in the application are the following.
\begin{enumerate}
    \item An \textbf{unregistered student} can:
    \begin{itemize}
        \item create an account
    \end{itemize}
    \item A \textbf{student} can:
    \begin{itemize}
        \item login/logout
        \item browse courses by course name
        \item see his starred course
        \item see and manage his booked meetings
        \item select a course in order to:
        \begin{itemize}
            \item view course details
            \item star that course
            \item enter the course chatroom
            \item book a meeting for that course
        \end{itemize}
    \end{itemize}
    \item A \textbf{professor} can:
    \begin{itemize}
        \item login/logout
        \item see and manage his booked meetings
        \item create a new course selecting specific time-slots
        \item delete a course
    \end{itemize}
    \item The \textbf{admin} can:
    \begin{itemize}
        \item login/logout
        \item create professor account
        \item browse users by username and ban them
    \end{itemize}
\end{enumerate}
\ \\
For what concerns \textbf{non-functional requirements} we have, \textbf{for the web-app}:
\begin{itemize}
    \item Concurrent service accesses management
    \item Strong consistency for users, courses and meetings data stored in MySQL DB
\end{itemize}
and, \textbf{for the chatroom}:
\begin{itemize}
    \item Concurrent service accesses management
    \item High service availability
    \item Fault tolerance to node faults
    \item Allow high horizontal scalability
    \item Eventual consistency for chatroom state data
\end{itemize}
\newpage
