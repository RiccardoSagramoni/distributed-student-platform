As we anticipated in the section \vref{subsubsec:erlang/Client-server communication}. there is a message exchange between the browser client and the cowboy websocket handler.\\
The browser client sends and receives messages via a \textbf{websocket} that is implemented through \textbf{Javascipt}.\\
This websocket was created by using the \textbf{WebSocket API}, which provides methods and events to handle websocket connections.\\
We matched these methods and events with functions that handled them and included additional functions to make the chatroom work properly.\\

The functions that invoke the \textbf{creation and closure of the websocket} are:
\begin{itemize}
    \item \textit{connect(\_logged\_user, \_course)}, which is responsible for \textbf{creating the websocket and associating each method with the function that handles it}. This function is \textbf{called with the onload} of the body of the corresponding html page;
    \item \textit{disconnect()}, which is responsible for \textbf{invoking the websocket connection closure} and the function that handles the latter. This function is \textbf{called with the onunload} of the body of the corresponding html page. \\
\end{itemize}

The functions that \textbf{handle the websocket connection} are:
\begin{itemize}
    \item \textit{openWebsocketConnection()}, it \textbf{starts the connection with websocket}, sending the login message and also call of the refreshWebsocketConnection() function;
    \item  \textit{closeWebsocketConnection()}, is invoked when the \textbf{connection with the websocket is closed} and also call of the stopRefreshWebsocketConnection(). In addition This function tries to create a new websocket connection \textbf{in case the previous connection is involuntarily closed}, for example, after the Erlang Node, to which it was connected, \textbf{crashes};
    \item  \textit{refreshWebsocketConnection()}, it executes an \textbf{automatic refresh of the websocket connection} every 10 sec by sending GET\_ONLINE\_USERS message through the websocket connection;
    \item \textit{stopRefreshWebsocketConnection()}, it \textbf{stops the automatic refresh of the websocket connection};
    \item \textit{sendObjectThroughWebsocket(isWebsocketConfigurationMessage, ...args)},\\ it \textbf{creates and sends a JSON object} through the websocket connection;
    \item \textit{receiveObjectThroughWebsocket(event)}, it \textbf{handles the receive event of a JSON object}, parse it and executes the action provided by the opcode parameter of the object;
    \item \textit{handleWebsocketError()}, it \textbf{handles the possible errors} with the websocket connection.
\end{itemize}
